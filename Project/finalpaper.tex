\documentclass[11pt]{article}
\usepackage{graphicx}    % needed for including graphics e.g. EPS, PS
\topmargin -1.5cm        % read Lamport p.163
\oddsidemargin -0.04cm   % read Lamport p.163
\evensidemargin -0.04cm  % same as oddsidemargin but for left-hand pages
\textwidth 16.59cm
\textheight 21.94cm 
%\pagestyle{empty}       % Uncomment if don't want page numbers
\parskip 7.2pt           % sets spacing between paragraphs
%\renewcommand{\baselinestretch}{1.5} % Uncomment for 1.5 spacing between lines
\usepackage{amsmath}
\usepackage{amsfonts}
\usepackage{amsthm}
\usepackage{verbatim}
\parindent 0pt		 % sets leading space for paragraphs
\author{by Erik Waingarten and Fermi Ma}
\title{Data Structures Final Project:\\
A Linear Programming Approach to Dynamic Optimality}

\newtheorem{theorem}{Theorem}
\newtheorem{lemma}[theorem]{Lemma}
\begin{document}         
\maketitle

\section{Introduction}

We consider the geometric interpretation of dynamic optimality, known as the points in the plane problem. The problem is defined as follows:

\vbox{
\noindent
\begin{quote}
Given a set $P$ of $n$ points $(x_i,y_i)$ in the $xy$ plane, no two on a common row or column, find the minimal point set $Q \supseteq P$ such that for any two points in $Q$ not on a common row or column, the rectangle they span contains another point in $Q$.
\end{quote}
}

Point sets $Q$ that satisfy the condition that any two points in $Q$ that span a rectangle contain another point in $Q$ are commonly referred to as \emph{arborally satisfied} point sets.

We consider an integer linear programming approach to the problem. It is natural to consider the grid of intersection points ``induced" by the set of points $P$: extend horizontal and vertical lines through each point in $P$ and restrict attention to the $O(n^2)$ intersections of these lines. This is because any points off the grid can be moved to points that are grid intersections.

We can assign an indicator variable to each grid point, where the variable is set to 1 if the point is included and is 0 otherwise. This gives the obvious objective of minimizing the sum of all variables, and it remains to translate the arboral satisfaction constraint into a system of linear constraints on the indicator variables. In this paper, we restrict our attention to linear programs that follow this model. While other linear programming approaches almost certainly exist, this sort of setup certainly feels the most natural.

\section{A First Attempt}

We set up the indicator variables as explained in the introduction, labeling them as $b_{jk}$, where $j$ and $k$ are labels on the grid intersection points (labelled so that $b_{00}$ is the lower left grid point, $b_{01}$ is the point directly above, and $b_{10}$ is the point directly to the right).

Consider the following set of constraints:

\begin{align}
b_{ij} + b_{nm} - \left(\left(\sum_{i \leq k \leq j}\sum_{n\leq l \leq m} b_{kl}\right) - b_{ij} - b_{nm}\right) &\leq 1  \hspace{.3in} \forall i<n, j<m \\
b_{im} + b_{nj} - \left(\left(\sum_{i \leq k \leq m}\sum_{n\leq l \leq j} b_{kl}\right) - b_{im} - b_{nj}\right)  &\leq 1  \hspace{.3in} \forall i<n, j<m
\end{align}

We claim that these constraints exactly capture arboral satisfaction. Constraints of type (1) correspond to all possible ``positive" rectangles, and constraints of type (2) correspond to all possible ``negative" rectangles. 

If we look at constraints of type (1), the vertices corresponding to variables $b_{ij}$ and $b_{nm}$ are present, those variables are set to 1 and sum to 2, and this forces the following term in parenthesis to be at least 1 in order for the constraint to be satisfied. The term in parenthesis corresponds to all points in the rectangle spanned by these vertices, and so forcing one of those variables to be 1 is equivalent to ensuring that the rectangle is satisfied. The constraints of type (2) work the same way.

If the two corner vertices are not present, the positive terms in the constraint are at most 1, and so the constraint is satisfied no matter what the term in parenthesis is.

However, the primary issue with writing constraints like this is that they are possibly quadratic in size. Two vertices may span a rectangle of $O(n^2)$ points, and it seems difficult to extract an $O(1)$ approximation scheme from such constraints.

\section{A Second Attempt: Reducing Constraint Size}

To set up linearly sized constraints, we need a simple lemma.

\begin{lemma}
Any arborally satisfied point set can be modified with $O(1)$ overhead so that all the rectangles are satisfied by grid points that lie on their edges.
\end{lemma}

\begin{proof} We consider the process of adding points to satisfy unsatisfied rectangles. Consider any unsatisfied negative rectangle, with vertices $v_1 = (0,0)$ and $v_2 = (a,b)$. Suppose it is optimal to satisfy this rectangle by filling in a non-edge point, at vertex $v_3 = (c,d)$ where $c < a$ and $d< b$.

We claim that instead of placing vertex $v_3$, we could have placed the two vertices $(c,b)$ and $(a,d)$. These two vertices clearly satisfy the original rectangle defined by $v_1$ and $v_2$, so the only worry now is that $v_3$ was placed ``optimally" to satisfy other rectangles that intersect this rectangle (and are defined by vertices that lie entirely outside this rectangle). But we note that these two new vertices $(c,b)$ and $(a,d)$ will satisfy any such rectangle that $v_3 = (c,d)$ might have satisfied. Thus, we can simply replace this vertex with these two vertices, and lose at most a factor of 2.

Note that a symmetric argument works for any given positive rectangle

\end{proof}

We claim that the following set of constraints captures arboral satisfaction: 

\begin{align}
b_{ij} + b_{nm} - \left(\sum_{l=i+1}^{n-1} (b_{lj}+b_{lm}) + \sum_{l=j+1}^{m-1} (b_{il} + b_{nl}) + b_{im} + b_{nj} \right) &\leq& 1 & \hspace{.3in} \forall i<n, j<m \\
b_{im} + b_{nj} - \left(\sum_{l=i+1}^{n-1} (b_{lj}+b_{lm}) + \sum_{l=j+1}^{m-1} (b_{il} + b_{nl}) + b_{ij} + b_{nm} \right) &\leq& 1 & \hspace{.3in} \forall i<n, j<m
\end{align}

Constraints of type (3) correspond to all possible ``positive" rectangles, and constraints of type (4) correspond to all possible ``negative" rectangles. This captures arboral satisfaction for the same reason that the constraints of types (1) and (2) do. The only difference is that these constraints only look at vertices on the edges of rectangles.

\section{Possible Approaches}

\subsection{Early Experiments}

Early experiments showed that an integrality gap existed between the ILP above and its relaxed LP.

\subsection{Max Flow}

Our first attempt at solving the above LP was to rewrite it as a completely equivalent maximum flow problem (or min cost max flow). We were unsuccessful at coming up with such a formulation, and we believe that no simple max flow formulations of this exact LP exist. A natural way of rewriting this problem as a max flow problem would almost certainly result in a flow problem with integral capacities, given that all coefficients in the constraints in the LP are integers. However, the Integral Flow Theorem would then guarantee that the optimal solution to this LP would be integral. This cannot be the case, since experimental evidence shows that there is an integrality gap strictly greater than 1.

\subsection{The Dual}

We can represent the linear program as a matrix, which will make it easier to take its dual:
\[ \min \hspace{.1in} [ \begin{array}{cccc} 1 & 1 & \dots & 1 \end{array} ] \left[\begin{array}{c} b_{11} \\ b_{12} \\ \vdots \\ b_{nn} \end{array} \right]\]
such that
\[ \left[ \begin{array}{c} \vdots \\ \text{ $P$ = positive rectangles } \\ \vdots \\ \text{ $N$ = negative rectangles } \\ \vdots \\ \text{ $I_n$ = equality } \end{array} \right] 
\left[\begin{array}{c} b_{11} \\ b_{12} \\ \vdots \\ b_{nn} \end{array} \right] 
\begin{array}{c} \\ \leq \\ \\ \\ \leq \\ \\ \\ = \end{array} 
\left[\begin{array}{c} 1 \\\\ 1 \\\\ \vdots \\\\ 1 \end{array} \right] \]

Where $P$ and $N$ are $\dbinom{n}{2}^2 \times n^2$ matrices, since there are that many positive and negative rectangles. And $I_n$ is the identity matrix. Furthermore, we require that all variables are positive. 

So we can take the dual:
\[ \max y^T \left[ \begin{array}{c} 1 \\ 1 \\ \vdots \\ 1 \end{array} \right] \]
such that
\[ \left[ \begin{array}{ccc} \vdots & \vdots & \vdots \\
					  P^T & N^T & I_n \\
					\vdots & \vdots & \vdots \end{array} \right] y
\leq  \left[ \begin{array}{c} 1 \\ 1 \\ \vdots \\ 1 \end{array} \right] \]
Where the first $n$ variables in $y$ can take on any value, and the rest must all be negative. 

This will corresponds to the following:

We will have a variable for each positive rectangle and each negative rectangle, denoted $b_{ij, lk}$ and a variable for each point that was set $a_{xy}$. Then for each point $xy$, if $xy$ is set, then we have the following constraint:
\[ a_{xy} + \sum_{\text{ rectangles where $xy$ on corner }} b_{ij, lk} - \sum_{\text{ rectangles where $xy$ on side }} b_{ij,lk} \leq 1 \]
If $xy$ is not set, then the $a_{xy}$ term is not present. We want to maximize
\[ \sum a_{xy} + \sum b_{ij,lk} \]
where $b_{ij,lk} \leq 0$. And the $a$'s are unconstrained. In some sense, we would like to set the values of the rectangles that have points two points on their corners as negative, since then we could raise $a_{xy}$ by the same amount in both corners and thus increase the objective.

\section{An Unbounded Integrality Gap}

We show that the linear programming relaxation will probably not yield an $O(1)$-approximation.

There are known instances of the problem that require $O(n\log n)$ additional points. On one of these instances, if you set all the neighboring points of set points to $0.5$. This is a feasible solution in the LP whose optimum is $O(n)$. Therefore, the integrality gap is $O(\log n)$. 

This means that typical rounding schemes won't work. 

\section{Extending the Method}

\section{Conclusions}

\end{document}
